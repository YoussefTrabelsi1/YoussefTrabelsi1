\documentclass{article}
\usepackage[utf8]{inputenc}

\title{\Huge Comptes rendus réunions n°1}
\author{Participants : Ridel Julien - Youssef Trabelsi \\ Durée : 2h30}
\date{16/11/2021}

\usepackage{natbib}
\usepackage{graphicx}

\begin{document}

\maketitle

\section{\huge Objectifs de la séance}

\Large Brainstorming autour du thème et recherche d’applications déjà existantes.

\section{\huge Brainstorming autour du thème} 

\Large Plusieurs idées ont été abordées. On à terminé par se mettre d’accord sur une idée globale de l’application : \\

\begin{itemize}

    \item \Large L’application devra permettre aux citoyens de proposer des idées qui seront soumises à une évaluation de la part de leurs concitoyens.
    
    \begin{itemize}
        \item \Large Chaque propositions sera validées ou non. Si elle est validée, cette idée est soumise au conseil municipal qui pourra décider soit d’agir directement, soit de consulter ses citoyens, soit de ne pas étudier le projet, je jugeant non pertinent.
    \end{itemize} 
    
    \item \Large L’application devra permettre au conseil municipal de consulter les citoyens sur divers thèmes, par le biais d’une question, sur le même principe du référendum.
    
    \item \Large L’application devra permettre aux utilisateurs de réaliser des sondages et des pétitions.

    \item Une fonctionnalité “bonus” à été aussi abordée. Les citoyens pourraient signaler des problèmes dans leur ville comme par exemple, signaler un dépôt sauvage d’ordures.
    
    \item \Large Une partie de l’application serait présente pour afficher les actualités de la ville ainsi que les propositions ayant affecté la vie de la ville.
    
    \item \Large Certaines idées de designs ont été apportés mais rien de concluant pour l’instant.
    
    \item \Large Chaque propositions/idées etc… aura un système de like/dislike comme les réseaux sociaux ainsi qu’une partie commentaire afin d’ouvrir des débats entre citoyens (bien sûr, un système de signalement devra sûrement être mis en place afin d’éviter les débordements et les conflits, ce qui serait contre productif). 
\end{itemize}

\section{\huge Recherche approfondie sur le Web d'applications déjà existante}

\begin{itemize}
\item \Large Une recherche approfondie sur le Web à permis de mettre en évidence différentes applications de CivicTech sur le marché.

\item \Large Début de la conception de l'État de l’Art  en comparant toutes les applications retenues avec des critères qu’il nous reste encore à pleinement identifier.
\end{itemize}

\end{document}
