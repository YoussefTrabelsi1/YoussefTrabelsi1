\documentclass{article}
\usepackage[utf8]{inputenc}

\title{\Huge Comptes rendus réunions n°2}
\author{Participants : Ridel Julien - Youssef Trabelsi - François Denes \\ Durée : 2h}
\date{17/11/2021}

\usepackage{natbib}
\usepackage{graphicx}

\begin{document}

\maketitle

\section{\huge Objectifs de la séance}

\Large Définir clairement nos envies pour l’application souhaitée, mettre en place un début de base de donnée et créer un style graphique pour notre application.

\section{\huge Définition de notre application: } 
\Large Plusieurs idées ont été abordées. On à terminé par se mettre d’accord sur une idée globale de l’application : \\

\begin{itemize}

    \item \Large L’application devra permettre aux citoyens de proposer des idées qui seront soumises à une évaluation de la part de leurs concitoyens.
    
    \item \Large Plusieurs idées ont été échangées pour au final converger vers une idée commune.
    
    \item \Large Nous avons donc décidé de nous orienter vers une application permettant aux citoyens et à l’administration de soumettre des idées à un vote. Suivant le résultat, la Mairie pourra prendre des dispositions afin de mettre en place l’idée abordée.

    \item Nous avons aussi déterminé un ensemble d’idées avec un ordre de priorité afin de pouvoir rendre en priorité les éléments principaux de l’application tout en nous laissant des possibilités d’ajout “bonus” comme par exemple, un système de commentaires sous chaque proposition afin de donner la possibilité de débattre.
\end{itemize}

\section{\huge Début de base de donnée :}

\begin{itemize}
\item \Large Après avoir clairement explicité nos envies sur l’application, nous avons commencé à proposer une première ébauche de base de données.

\item \Large Parler de base de données, à ce stade est un peu trop avancé. Nous avons surtout explicité les parties importantes qu’il faudrait implémenter dans une base de donnée afin d’assurer le bon fonctionnement de notre application.
\end{itemize}

\section{\huge Création du style graphique :}

\begin{itemize}
\item \Large Après un long moment à expliciter un schéma bref sur une probable base de données, nous avons décidé de commencer une recherche de style graphique pour conclure notre réunion.

\item \Large Quelques schémas des principales pages ont été dessiné ce qui nous a donné un avis sur la complexité d’écriture des programmes que nous devons faire afin de suivre le mieux possible nos schémas.
\end{itemize}
\end{document}
