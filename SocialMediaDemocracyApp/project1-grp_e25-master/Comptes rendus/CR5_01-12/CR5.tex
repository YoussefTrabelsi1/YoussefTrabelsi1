\documentclass{article}
\usepackage[utf8]{inputenc}

\title{\Huge Comptes rendus réunions n°5}
\author{Participants : Ridel Julien - Youssef Trabelsi - François Dénès \\ Durée : 2h}
\date{01/12/2021}

\usepackage{natbib}
\usepackage{graphicx}

\begin{document}

\maketitle

\section{\huge Objectifs de la séance }

\Large Résumer les tâches déjà effectuées et mettre en place une matrice RACI

\section{\huge Résumé des tâches déjà effectuées}

\begin{itemize}

    \item \Large Tâches déjà réalisées :
    
    \begin{itemize}
        \item \Large Mise en accord des idées et décision de la marche à suivre (Faite par le groupe)
        \item \Large  Prévisions de ce qui peut être fait dans le temps imparti en prenant en compte les capacités du groupe (Faite par François Dénès)
        \item \Large Les premières maquettes visuelles de l'application (Faite par Julien Ridel)
        \item \Large Mise en place d'un début de schéma relationnel sur papier qu'il faudrat améliorer et mettre au format draw.io (Faite par Youssef Trabelsi)
        \item \Large  Validation de l'idée auprès de M. Festor
    \end{itemize} 
\end{itemize}

\section{\huge Mise en place de la matrice RACI}

\Large Nous avons réalisé une matrice RACI en essayant de prendre en compte les capacités propres de toutes les personnes du groupe ainsi que leurs préférences sur les différentes tâches à réaliser. Cette matrice a été réalisée en gardant à l'esprit que notre groupe serait flexible sur l'assignation des tâches. Ce qui veut dire que si une personne a une idée pour réaliser une tâche d'un autre, il peut la faire mais doit obtenir l'accord de la personne en charge de cette tache.
Voici ci-dessous la matrice RACI réalisée. (Le png ci dessous est disponible dans le dossier du compte rendu) 

\centering \includegraphics[scale=0.5]{MATRICE_RACI.png} \\
\Large I : Informé / R : Réalise / A : Autorité
\end{document}
