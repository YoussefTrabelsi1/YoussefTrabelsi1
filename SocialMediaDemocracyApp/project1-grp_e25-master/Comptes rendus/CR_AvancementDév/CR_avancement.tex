\documentclass{article}
\usepackage[utf8]{inputenc}

\title{\Huge Comptes rendus réunions n°7}
\author{Participants : Ridel Julien - Youssef Trabelsi - François Dénès \\ Durée : 2h}


\usepackage{natbib}
\usepackage{graphicx}

\begin{document}

\maketitle

\section{\huge Objectifs de ce compte rendu }

\Large Ce document a pour but de répertorier les différentes discussions opérées par les membres du groupe afin de témoigner de l'avancement du projet et des différentes difficultés rencontrées tout au long du développement de l'application 

\section{\huge Point du 21 décembre 2021 par Ridel Julien}

\begin{itemize}

    \item \Large Lors de cette journée, la mise en place des éléments primordiaux de Flask a été effectuée (Templates, static, App.py). Du temps a été pris pour faire le lien entre html et css. Après cela, un début de page pour permettre la création des posts a été codée sans certains éléments, comme la possibilité de télécharger des images sur l'application et le style css de l'ensemble. Tout semblait fonctionner, il était possible de voir les posts créé directement sur la page principale de notre site. A ce moment là, il était impossible de consulter le contenu des posts.
    Pour faire fonctionner le tout, une première base de données a été mise en place en utilisant sqlite3.

\end{itemize}

\section{\huge Point du 22 décembre 2021 par Ridel Julien}

\begin{itemize}

    \item \Large Les éléments codés la veille ont été repris afin de rajouter des options et tout a été push sur le répertoire git.

\end{itemize}

\section{\huge Point du 23 décembre 2021 par Ridel Julien}

\begin{itemize}

    \item \Large Lors de cette journée, tout ce qui à été fait la veille a été repris. Le style des différentes pages a été mis en place. Un éditeur de texte "complet" ainsi que la fonctionnalité d'ajout d'image a été ajoutés à l'éditeur de posts. Les images à ce moment là, doivent être ajoutées via un lien web.
    Enfin, un début de page pour consulter le contenu des posts à été ajoutée. Le style a pris beaucoup de temps afin de pouvoir agencer tous les éléments du post, mais à ce moment là, le contenu du post n'était pas pleinement consultable.
    
\end{itemize}

\section{\huge Point du 24 décembre 2021 par Dénès François}

\begin{itemize}

    \item \Large  Les grands axes du controleur ont été mis en place en utilisant Flask. Les routes vers les pages commencées et prévues ont été mises en place, ainsi qu'un certain nombre de fonctions utilisataires, permettant par exemple de trier et filtrer les posts selon les critères définis. L'implémentation du schéma relationnel en SQL au travers de la création de tables a également été commencé au cours des semaines et jours précédents.
    
\end{itemize}

\section{\huge Point du 27 décembre 2021 par Dénès François}

\begin{itemize}

    \item \Large L'affichage des informations des posts au sein d'une page dédiée a été implémenté. Les informations propres au post telles que son titre, son contenu ou encore son auteur ont été ajoutés dans un premier temps. Les informations plus complexes telles que les avis et les commentaires seront ajoutés par la suite car plus compliquées à mettre en oeuvre et nécessitant encore une réflexion quant à la façon d'implémenter ces fonctionnalités.
                 La première version de la base de donnée a été complétée, avec l'ajout de vues simplifiant l'accès et la modification des données depuis la partie web. La base de donnée pourra encore éventuellement être légèrement modifiée si le besoin s'en fait sentir pour faciliter l'intégration d'une fonctionnalité.
    
\end{itemize}

\section{\huge Point du 28 décembre 2021 par Ridel Julien, Dénès François et réunion entre Dénès François et Trabelsi Youssef}

\begin{itemize}

    \item \Large Ridel Julien a commencé l'ajout des différentes réunions effectuées. Les notes prises lors de ces réunions ont dû être organisées ainsi que codée en LaTeX. Seulement 2 ont été ajoutées, les autres suivront.
    
    \item \Large Dénès François a continué l'affichage des posts, avec l'implémentation de l'affichage des avis donnés sur le commentaire, et l'ajout ou la modification de ceux-ci par l'utilisateur.

    \item \Large Lors de la réunion entre Dénès François et Trabelsi Youssef (Ridel Julien ne pouvait être présent à cause d'un impératif), une présentation de l'avancement du projet a été effectuée. Les membres du groupe ont par la suite exprimés les tâches sur lesquelles ils étaient en train de travailler afin d'avoir une vision globale sur l'avancement du projet et sur ce qui restait à faire.
    
\end{itemize}

\section{\huge Point du 30 décembre 2021 par Dénès François et Trabelsi Youssef}

\begin{itemize}

    \item \Large Youssef Trabelsi a commencé à travailler sur la fonction d'analyse des commentaires. Tout d’abord, il a pensé à implémenter une fonction qui permet la correction (grammaticale) automatique des commentaires afin de les analyser. Mais il s’est avéré que cette tâche n’était pas si simple à exécuter à cause de la possibilité d'avoir un très grand nombre de fautes. Il a donc supposé que les commentaires soient grammaticalement corrects. Il a affirmé que c’était difficile au début car il n’a pas trouvé ce qu’il voulait sur internet pour avoir une liste connue des mots positifs ou négatifs. D’après lui, la recherche d'un dictionnaire de ce type était si difficile car c’est un produit qui doit être acheté. \\
    Il a eu alors l’idée de regrouper tous les adjectifs/noms/verbes positifs et négatifs les plus utilisées et d’utiliser les règles de la langue française. (Par exemple le suffixe -ard permet souvent de construire des termes péjoratifs comme chauffard)\\
    Il a même pensé aux négations dans sa fonction pour éviter la fausse interprétation du commentaire.\\
    Pour cela il a tout d’abord créé une fonction prenant tous les mots du commentaire et les analyses (s’il est positif, négatif, ou une forme de négation) d’où sa fonction qui permet d'analyser les commentaires.
    
    \item \Large Dénès François a ajouté les fonctions de connexion et déconnexion. Les filtres et tris développés précédemment ont également été ajoutés à la page d'accueil affichant la liste des posts. L'ajout des commentaires et de leurs réponses vont être ajoutés au cours des jours suivant.
    
\end{itemize}

\section{\huge Point du 31 décembre 2021 par Ridel Julien}

\begin{itemize}

    \item \Large Ajout du document contenant tous les différents points d'avancements du développement ainsi qu'un début de document pour le rapport de projet.
    
\end{itemize}

\section{\huge Point du 1 décembre 2021 par toute l'équipe}

\begin{itemize}

    \item \Large L'équipe de projet s'est réunie pour faire une réunion afin de faire le point sur ce qui a été fait et sur ce qu'il restait à faire. A la fin de cette réunion, l'équipe a pu constater que l'application était majoritairement opérationnelle. Les tâches restante à faire vont surement être implémentées dans les jours qui suivent.
    
\end{itemize}

\section{\huge Point du 3 janvier 2022 par Trabelsi Youssef}

\begin{itemize}

    \item \Large Trabelsi Youssef a ajouté un début de page permettant de visionner le profil de l'utilisateur connecté.
    
\end{itemize}

\end{document}
